% At first is the PREAMBLE of the Document
% Mention in preamble the :-
%           1. Type Of Document
%           2. Document Language
%           3. Packages t be used Later


% Below is the document of article type.
% The font is kept to 12pt [10pt is default]
% The layout orientation is a4paper
% utf8 is the encoding done

\documentclass[12pt, a4paper ]{article}
\usepackage[utf8]{inputenc}

% Adding Other details
\title{Lorem Ipsum}
\author{DAIM KHAN}
% For date you can enter of can use `\today` command too.
\date{April 2021}


\begin{document}

%Above mentioned data can now be printed on document in Title
\maketitle
\tableofcontents

% Abstract

\begin{abstract}

Lorem Ipsum is simply dummy text of the printing and typesetting industry. Lorem Ipsum has been the industry's standard dummy text ever since the 1500s, when an unknown printer took a galley of type and scrambled it to make a type specimen book. It has survived not only five centuries, but also the leap into electronic typesetting, remaining essentially unchanged. 

\end{abstract}


% TEXT FORMATTING
%               1. For Bold-  \textbf{...}
%               2. For Italic-  \textit{...}
%               3. For underline-  \underline{...}
%               4. Emph: it italicizes the selected text if rest                      text is normal and do vice-versa

\newline
Some of the \textbf{greatest}
discoveries in \underline{science} 
were made by \textbf{\textit{accident}}.


Some of the greatest \emph{discoveries} 
in science 
were made by accident.

\textit{Some of the greatest \emph{discoveries} 
in science 
were made by accident.}

\textbf{Some of the greatest \emph{discoveries} 
in science 
were made by accident.}





% LISTS


% Unordered List
% To start a list- \begin{...} and to end a list- \end{...}

\begin{itemize}
  \item This is the first item.
  \item This is the Second item.
\end{itemize}



% Ordered List
% Same as Unordered, just instead of itemize we use enumerate 

\begin{enumerate}
  \item This is the 1 item.
  \item This is the 2 item.
  \item This is the 3 item.
  \item This is the 4 item.
\end{enumerate}


% Some Maths
Subscripts in math mode are written as $a_b$ and superscripts are written as $a^b$. These can be combined an nested to write expressions such as

\[ T^{i_1 i_2 \dots i_p}_{j_1 j_2 \dots j_q} = T(x^{i_1},\dots,x^{i_p},e_{j_1},\dots,e_{j_q}) \]
 
We write integrals using $\int$ and fractions using $\frac{a}{b}$. Limits are placed on integrals using superscripts and subscripts:

\[ \int_0^1 \frac{dx}{e^x} =  \frac{e-1}{e} \]

Lower case Greek letters are written as $\omega$ $\delta$ etc. while upper case Greek letters are written as $\Omega$ $\Delta$.

Mathematical operators are prefixed with a backslash as $\sin(\beta)$, $\cos(\alpha)$, $\log(x)$ etc.


% FORMATTING

\chapter{First Chapter}

\section{Introduction}

This is the first section.

Lorem  ipsum  dolor  sit  amet,  consectetuer  adipiscing  
elit.   Etiam  lobortisfacilisis sem.  Nullam nec mi et 
neque pharetra sollicitudin.  Praesent imperdietmi nec ante. 
Donec ullamcorper, felis non sodales...

\section{Second Section}

Lorem ipsum dolor sit amet, consectetuer adipiscing elit.  
Etiam lobortis facilisissem.  Nullam nec mi et neque pharetra 
sollicitudin.  Praesent imperdiet mi necante...

\subsection{First Subsection}
Praesent imperdietmi nec ante. Donec ullamcorper, felis non sodales...

\section*{Unnumbered Section}
Lorem ipsum dolor sit amet, consectetuer adipiscing elit.  
Etiam lobortis facilisissem




% TABLES


% Simple Table

\begin{center}
\begin{tabular}{ c c c }
 cell1 & cell2 & cell3 \\
 cell4 & cell5 & cell6 \\  
 cell7 & cell8 & cell9    
\end{tabular}
\end{center}


%Table Formatting

\begin{center}
\begin{tabular}{ |c|c|c| } 
 \hline
 cell1 & cell2 & cell3 \\ 
 cell4 & cell5 & cell6 \\ 
 cell7 & cell8 & cell9 \\ 
 \hline
\end{tabular}
\end{center}



% Advance Table

Table \ref{table:data} is an example of referenced \LaTeX{} elements.

\begin{table}[h!]
\centering
\begin{tabular}{||c c c c||} 
 \hline
 Col1 & Col2 & Col2 & Col3 \\ [0.5ex] 
 \hline\hline
 1 & 6 & 87837 & 787 \\ 
 2 & 7 & 78 & 5415 \\
 3 & 545 & 778 & 7507 \\
 4 & 545 & 18744 & 7560 \\
 5 & 88 & 788 & 6344 \\ [1ex] 
 \hline
\end{tabular}
\caption{Table to test captions and labels}
\label{table:data}
\end{table}


Combining lines, circles and text
\setlength{\unitlength}{0.8cm}
\begin{picture}(12,4)
\thicklines
\put(8,3.3){{\footnotesize $3$-simplex}}
\put(9,3){\circle*{0.1}}
\put(8.3,2.9){$a_2$}
\put(8,1){\circle*{0.1}}
\put(7.7,0.5){$a_0$}
\put(10,1){\circle*{0.1}}
\put(9.7,0.5){$a_1$}
\put(11,1.66){\circle*{0.1}}
\put(11.1,1.5){$a_3$}
\put(9,3){\line(3,-2){2}}
\put(10,1){\line(3,2){1}}
\put(8,1){\line(1,0){2}}
\put(8,1){\line(1,2){1}}
\put(10,1){\line(-1,2){1}}
\end{picture}



\end{document}
